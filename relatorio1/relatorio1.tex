% Created 2017-08-14 Seg 18:01
\documentclass[11pt]{article}
\usepackage[utf8]{inputenc}
\usepackage[T1]{fontenc}
\usepackage{fixltx2e}
\usepackage{graphicx}
\usepackage{longtable}
\usepackage{float}
\usepackage{wrapfig}
\usepackage{soul}
\usepackage{textcomp}
\usepackage{marvosym}
\usepackage{wasysym}
\usepackage{latexsym}
\usepackage{amssymb}
\usepackage{hyperref}
\tolerance=1000
\providecommand{\alert}[1]{\textbf{#1}}

\title{relatorio1}
\author{CCSL}
\date{\today}
\hypersetup{
  pdfkeywords={},
  pdfsubject={},
  pdfcreator={Emacs Org-mode version 7.9.3f}}

\begin{document}

\maketitle

\setcounter{tocdepth}{3}
\tableofcontents
\vspace*{1cm}

\section{Movimento1}
\label{sec-1}


Teoricamente, para uma velocidade de 300 graus/segundo == 5.23 rad/segundo, dado que o raio da roda é 2.8cm, a velocidade em cm é: 14.66cm/segundo. 

Em nosso experimento, o robo andou por 2.5 segundos. Na teoria ele deveria andar 36.6cm. Nos experimentos obtivemos os seguintes resultados:
pelos cálculos utilizando o tacômetro, 36.7 cm; pela trena, 36cm.

A variação no eixo y foi mínima de 0.07 cm. O esperado seria
0 cm, o que avaliamos ser um bom resultado.

A variação no ângulo foi de 0.003 radianos. O que também é muito bom.
\section{Movimento2}
\label{sec-2}

\end{document}
